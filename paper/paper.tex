\documentclass[11pt, fleqn]{article}
\usepackage[english]{babel}
\usepackage{amsmath}

\usepackage[linesnumbered]{algorithm2e}


\usepackage{natbib}
\setlength{\parskip}{\baselineskip}%
\setlength{\parindent}{0pt}%

\title{Spit-Apply-Combine with Collapsing Groups}
\author{Mark P.J. van der Loo}


\begin{document}
\maketitle

Let $U$ be a finite population, and let $A$ be a finite set with $|A|\leq |U|$.
Any function 
\begin{equation}
f: U\to A,
\end{equation}
defines a partition on $U$, in the sense that two elements are equivalent if they
map to the same element of $A$. In the rest of the discussion it will be useful to 
also define the formal inverse $f^{-1}:A\to 2^U$, such that $f^{-1}(a)$ is the set of 
elements in $U$ that maps to $a$. In notation:
\begin{equation}
f^{-1}(a) \equiv \{u\in U: f(u)=a\}.
\end{equation}

\begin{algorithm}[H]
\SetKwInOut{Input}{Input}\SetKwInOut{Output}{Output}
\Input{$U$, $\phi: 2^U\to X$, a partition $(f,A)$}
\Output{The value of $\phi$ for every part of $U$ induced by $(f,A)$, represented
       as a set pairs $(a,x)\in A\times X$. }

$R = \{\}$\;
\While{$U\not=\{\}$}{
Choose any $u\in U$\;
$d = (f^{-1}\circ f) (u)$\;
$U = U - d$\;
$ R= R\cup (f(u), \phi(d)$)\;
}

\end{algorithm}


\bibliographystyle{plainnat}
\bibliography{vanderloo}
\end{document}


